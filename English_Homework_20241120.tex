\documentclass[11pt,a4paper,titlepage]{jsarticle}
%
\usepackage{amsmath,amssymb}
\usepackage{bm}
\usepackage{ascmac}
\usepackage{empheq}
\usepackage[dvipdfmx]{graphicx}
\usepackage[dvipdfmx]{color}
\usepackage{float}
\usepackage{siunitx}
\usepackage{enumerate}
\usepackage{booktabs}
\usepackage{subcaption}
\usepackage{autobreak}
\usepackage{longtable}
\usepackage{listings}
%
\SetSymbolFont{letters}{normal}{OT1}{cmr}{m}{n}
\SetMathAlphabet{\mathnormal}{normal}{OT1}{cmr}{m}{n}
%
\setlength{\textwidth}{\fullwidth}
\setlength{\textheight}{40\baselineskip}
\addtolength{\textheight}{\topskip}
\setlength{\voffset}{-0.2in}
\setlength{\topmargin}{0pt}
\setlength{\headheight}{0pt}
\setlength{\headsep}{0pt}
%
\graphicspath{{./figure/}}
%
\everymath{\displaystyle}
%
\makeatletter
\def\@maketitle{
  \begin{flushright}
    {\large \@date}
  \end{flushright}
  \par\vskip 1.5em
  \begin{center}
    {\LARGE \@title \par}
  \end{center}
  \par
  \begin{flushright}
    {\large \@author}
  \end{flushright}
  \par\vskip 1.5em
}
\makeatother
%
\makeatletter
\newcommand{\figcaption}[1]{\def\@captype{figure}\caption{#1}}
\newcommand{\tblcaption}[1]{\def\@captype{table}\caption{#1}}
\makeatother
%
\newcommand{\divergence}{\mathrm{div}\,}  %ダイバージェンス
\newcommand{\grad}{\mathrm{grad}\,}  %グラディエント
\newcommand{\rot}{\mathrm{rot}\,}  %ローテーション
\newcommand{\const}{\mathrm{const.}\,} %一定
%
\title{}
\author{}
\date{\today}
%
\begin{document}

\setlength\lineskiplimit{20pt}
\setlength\lineskip{20pt}

\section*{Homework 2024-11-20}

\section{品詞決定}

以下の例文で使われているすべての単語について、品詞を決定して記せ。

\begin{align}
  &This ~ is ~ truly ~ a ~ legendary ~ diamond \text{.}\\
  &Experienced ~ craftsmen ~ masterfully ~ create ~ exquisite ~ silverwork \text{.}\\
  &She ~ is ~ clumsy ~ with ~ her ~ hands ~ and ~ often ~ makes ~ very ~ bad ~ food \text{.}\\
  &He ~ proposes ~ expanding ~ nuclear ~ power ~ generation ~ from ~ an ~ energy ~ security ~ perspective \text{.}\\
  &Many ~ high ~ school ~ students ~ in ~ science ~ courses ~ choose ~ geography ~ for ~ social ~ studies \text{.}\\
  &
  \begin{aligned}
    &While ~ American ~ rockets ~ have ~ only ~ a ~ few ~ large ~ boosters\text{,} ~
    Soviet ~ rockets ~ have ~ many\\ &small ~ boosters\text{.}
  \end{aligned}
\end{align}

\section{英文和訳}

先ほどの例文をすべて和訳せよ。

\section{和文英訳}

以下の和文をすべて英訳せよ。
翻訳機は使ってはならないが、和英辞典とGrammarlyは使用してもよい。

\begin{enumerate}
  \item 証人だけが真実を知っている。\\
  \item イギリス人は皮肉を好む。\\
  \item 保守思想がヨーロッパを支配する。\\
  \item その計画は物理的に不可能だ。\\
  \item 人は都合の悪い情報を無視する。\\
  \item 天ぷらは日本の伝統的な料理だ。
\end{enumerate}

\end{document}