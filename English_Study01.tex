\documentclass[11pt,a4paper,titlepage]{jsarticle}
%
\usepackage{amsmath,amssymb}
\usepackage{bm}
\usepackage{ascmac}
\usepackage{empheq}
\usepackage[dvipdfmx]{graphicx}
\usepackage[dvipdfmx]{color}
\usepackage{float}
\usepackage{siunitx}
\usepackage{enumerate}
\usepackage{booktabs}
\usepackage{subcaption}
\usepackage{autobreak}
\usepackage{longtable}
\usepackage{listings}
%
\SetSymbolFont{letters}{normal}{OT1}{cmr}{m}{n}
\SetMathAlphabet{\mathnormal}{normal}{OT1}{cmr}{m}{n}
%
\setlength{\textwidth}{\fullwidth}
\setlength{\textheight}{40\baselineskip}
\addtolength{\textheight}{\topskip}
\setlength{\voffset}{-0.2in}
\setlength{\topmargin}{0pt}
\setlength{\headheight}{0pt}
\setlength{\headsep}{0pt}
%
\graphicspath{{./figure/}}
%
\everymath{\displaystyle}
%
\makeatletter
\def\@maketitle{
  \begin{flushright}
    {\large \@date}
  \end{flushright}
  \par\vskip 1.5em
  \begin{center}
    {\LARGE \@title \par}
  \end{center}
  \par
  \begin{flushright}
    {\large \@author}
  \end{flushright}
  \par\vskip 1.5em
}
\makeatother
%
\makeatletter
\newcommand{\figcaption}[1]{\def\@captype{figure}\caption{#1}}
\newcommand{\tblcaption}[1]{\def\@captype{table}\caption{#1}}
\makeatother
%
\newcommand{\divergence}{\mathrm{div}\,}  %ダイバージェンス
\newcommand{\grad}{\mathrm{grad}\,}  %グラディエント
\newcommand{\rot}{\mathrm{rot}\,}  %ローテーション
\newcommand{\const}{\mathrm{const.}\,} %一定
%
\newcommand{\us}[2]{\underset{#1}{\vphantom{|_|}#2}}
\newcommand{\ub}[1]{\underbrace{#1}}
\newcommand{\ur}[1]{\underbracket{#1}}
%
\title{}
\author{}
\date{\today}
%
\begin{document}

\section{品詞}

\subsection{品詞の分類}

\subsubsection{名詞(Noun)}
名詞はモノやコトを表す品詞であり、文構造の骨格にかかわってくる品詞である。
文構造の中ではS(主語)、O(目的語)を主に担い、場合によってはC(補語)を担うこともある。
また、複数の単語が集まって名詞句や名詞節を形成すると、その塊で名詞としての役割を担うことがある。\\
e.g., Apple, School, Pen

\subsubsection{動詞(Verb)}
動作や状態などを表す品詞であり、文構造の骨格にかかわってくる品詞である。
文構造の中ではV(述語)を担う。\\
e.g., take, have, live

\subsubsection{助動詞(Auxiliary verb)}
文字通り動詞の意味を補助する役割の品詞である。
動詞とセットで意味を成す。\\
e.g., can, will, must

\subsubsection{形容詞(Adjective)}
名詞(名詞句、名詞節)を修飾する品詞である。
必ず紐づいている名詞がある。
通常の限定用法では文構造の骨格にかかわってこないが、直接S(主語)を叙述するような叙述用法ではC(補語)を担う。\\
e.g., beautiful, cool, high

\subsubsection{副詞(Adverb)}
名詞以外(主に動詞や文全体)を修飾する品詞である。
文構造の骨格にかかわってくることはない。
ある意味、ほかの分類に当てはまらないものが副詞と分類されていると言ってもよい。\\
e.g., always, carefully, just

\subsubsection{接続詞(Conjunction)}
文と文をつないだり、節を導いたりする品詞である。\\
e.g., and, but, if

\subsubsection{前置詞(Preposition)}
文字通り名詞の前に置いて関係を示す品詞である。\\
e.g., in, on, of

\subsubsection{冠詞(Article)}
名詞の前に置いて、その名刺が特定のものかそうでないかを示す品詞である。\\
e.g., a, an, the

\subsubsection{間投詞(Interjection)}
「わぁ」のように呼びかけを表したりする。\\
e.g., wow, oh, ah

\subsection{品詞決定の重要性}
以下の表を見てみよう。

\begin{table}[h]
  \centering
  \begin{tabular}{ccl}
    \hline
    品詞 & 意味 & \multicolumn{1}{c}{例文} \\
    \hline \hline
    名詞 & 背中、背後 & 
    \begin{tabular}{l}
      There is a bug on my back.\\
      背中に虫がついている。
    \end{tabular} \\
    動詞 & 後援する & 
    \begin{tabular}{l}
      I back this politician.\\
      私はこの政治家を後援している。
    \end{tabular} \\
    形容詞 & 後ろの & 
    \begin{tabular}{l}
      I entered by the back door.\\
      私は裏口から中へ入った。
    \end{tabular} \\
    副詞 & 後ろに & 
    \begin{tabular}{l}
      I came back.\\
      私は戻ってきた。
    \end{tabular} \\
    \hline
  \end{tabular}
\end{table}

例ではbackを挙げたが、名詞・動詞・形容詞・副詞など、使い方が複数ある英単語は枚挙にいとまがない。
例文のように単純明快な文章であればよいが、複雑な文章では、語順、近くの単語の形、文脈上のつながりなどを手掛かりにして、各単語の品詞を決定させなければならない。
品詞の決定(特に動詞)を誤ると、文構造の把握は当然であるが、文の概要の把握にも失敗してしまう。
入試問題で見かける難しい文章というのは、品詞決定しずらい単語を使ったり、入れ子構造を使ったり、転置や省略を多用したりして、巧みに難解な文章を作り上げている。
このような文構造のトリックを看破できるようになれば、英語はもはや敵ではない。

\section{要素と文型}

\subsection{要素}

\subsubsection{主語S(Subject)}
主語の要素を担うのは名詞(名詞句・名詞節)のみである。
どの文型にも必須の要素である。

\subsubsection{述語V(Verb)}
述語の要素を担うのは動詞(動詞句)のみである。
どの文型にも必須の要素である。

\subsubsection{目的語O(Object)}
目的語の要素を担うのは名詞(名詞句・名詞節)のみである。
SVO、SVOO、SVOCに必要な要素である。

\subsubsection{補語(Complement)}
補語の要素を担うのは名詞(名詞句・名詞節)と形容詞(形容詞句・形容詞節)のみである。
SVC、SVOCに必要な要素である。

\subsubsection{修飾語(Modifier)}
修飾語とは、上記の4要素以外のものである。
修飾語の要素を担うのは主に副詞(副詞句・副詞節)である。
文構造にはかかわらない要素なので、文の中にいくつあってもよいし、逆になくてもよい。
よって、5文型のどの構成要素でもない。

\begin{table}[h]
  \centering
  \begin{tabular}{ll}
    \hline
    \multicolumn{1}{c}{要素} & \multicolumn{1}{c}{品詞}\\
    \hline \hline
    主語S & 名詞\\
    述語V & 動詞\\
    目的語O & 名詞\\
    補語C & 名詞・形容詞\\
    修飾語M & 副詞\\
    \hline
  \end{tabular}
\end{table}

\subsection{文型}

\subsubsection{第1文型: SV}
主語Sと述語Vで構成される文。
例えば、純粋にSとVだけで構成されている文章としては、以下のようなものがある。

\begin{equation}
  \ub{\us{代名詞}{I}}_S ~ \ub{\us{動詞}{walk}}_V \text{.}
\end{equation}

他にも、副詞が修飾語Mとして付くようなケースもあるが、修飾語Mは文構造にかかわらないため、同じくSVの構造になる。

\begin{equation}
  \ub{\us{代名詞}{I}}_S ~ \ub{\us{動詞}{walk}}_V ~ \ub{\us{副詞}{slowly}}_M \text{.}
\end{equation}

前置詞に導かれている名詞がある場合は要注意である。
前置詞に導かれている名詞は副詞句を形成するので、修飾語となる。
つまり、文構造の骨格からは省かれる。

\begin{equation}
  \ub{\us{代名詞}{I}}_S ~ \ub{\us{動詞}{walk}}_V ~ \ub{\ur{\us{前置詞}{to} ~ \us{名詞}{school}}_{副詞句}}_M \text{.}
\end{equation}

\subsubsection{第2文型: SVC}
主語Sと述語Vと補語Cで構成される文。
「主語S=補語C」の関係が成り立っている。
補語Cは名詞の場合と形容詞の場合がある。

\begin{equation}
  \ub{\us{代名詞}{This} ~ \us{名詞}{flower}}_S ~ \ub{\us{動詞}{is}}_V ~ \ub{\us{前置詞}{a} ~ \us{名詞}{rose}}_C \text{.}
\end{equation}

\begin{equation}
  \ub{\us{代名詞}{This} ~ \us{名詞}{flower}}_S ~ \ub{\us{動詞}{is}}_V ~ \ub{\us{形容詞}{red}}_C \text{.}
\end{equation}

確かに、$This ~ flower = a ~ rose$と$This ~ flower = red$の関係が成り立っていることがわかる。
SVのときと同様に、修飾語Mがつくケースがある。
繰り返しになるが、修飾語Mは文構造にかかわらないため、同じくSVCの構造になる。

\begin{equation}
  \ub{\us{代名詞}{This} ~ \us{名詞}{flower}}_S ~ \ub{\us{動詞}{is}}_V ~ \ub{\us{副詞}{probably}}_M ~ \ub{\us{前置詞}{a} ~ \us{名詞}{rose}}_C \text{.}
\end{equation}

\subsubsection{第3文型: SVO}
主語Sと述語Vと目的語Oで構成される文。
「主語Sが目的語Oを動詞Vする」の関係が成り立っている。
基本的には動詞Vの直後に目的語Oが来るので、動詞の直後に名詞の役割をするものが続いているとき、必ずそれは目的語Oである。
一般的に、日本語で「○○を」となるときは目的語をとってSVOの構文になると言われている。
おおよそは当てはまっているが、例外は多々あるのでこの覚え方は要注意。

\begin{equation}
  \ub{\us{代名詞}{I}}_S ~ \ub{\us{動詞}{drink}}_V ~ \ub{\us{形容詞}{cold} ~ \us{名詞}{juice}}_O \text{.}
\end{equation}

\subsubsection{第4文型: SVOO}
主語Sと述語Vと目的語$O_1$と$O_2$で構成される文。
「主語Sが目的語$O_1$に目的語$O_2$を動詞Vする」の関係が成り立っている。
基本的には動詞Vの直後に目的語$O_1$が来て、その直後に目的語$O_2$が来る。

\begin{equation}
  \ub{\us{代名詞}{I}}_S ~ \ub{\us{動詞}{give}}_V ~ \ub{\us{名詞}{you}}_{O_1} ~ \ub{\us{冠詞}{a} ~ \us{名詞}{gift}}_{O_2} \text{.}
\end{equation}

SVOOの構文はSVOの形に書き換えることができる。

\begin{equation}
  \ub{\us{代名詞}{I}}_S ~ \ub{\us{動詞}{give}}_V ~ \ub{\us{冠詞}{a} ~ \us{名詞}{gift}}_O ~ \ub{\ur{\us{前置詞}{to} ~ \us{名詞}{you}}_{副詞句}}_{M} \text{.}
\end{equation}

意味は同じく「私はあなたにプレゼントをあげる」というものだ。
文法上は、$O_1$の頭に前置詞$to$が付いて修飾語Mになり、$O_2$が目的語Oとなっている。
$O_1$の頭に付くのは主に$to$だが、$for$などのほかの前置詞が付く場合もある。
細かい部分は後に取り上げる。

\subsubsection{第5文型: SVOC}
主語Sと述語Vと目的語Oと補語Cで構成される文。
SVCのときと近いが、今度は「目的語O=補語C」の関係が成り立っている。
意味上は「主語Sが目的語Oを補語Cに動詞Vする」の関係が成り立っている。
動詞の直後に名詞の役割をするものが2連続で来ているときは、SVOOかSVOCの構文を疑おう。
SVCのときと同様に、補語Cは名詞の場合と形容詞の場合がある。

\begin{equation}
  \ub{\us{代名詞}{They}}_S ~ \ub{\us{動詞}{call}}_V ~ \ub{\us{名詞}{me}}_O ~ \ub{\us{名詞}{John}}_{C} \text{.}
\end{equation}

\begin{equation}
  \ub{\us{代名詞}{I}}_S ~ \ub{\us{動詞}{keep}}_V ~ \ub{\us{代名詞}{my} ~ \us{名詞}{room}}_O ~ \ub{\us{形容詞}{clean}}_{C} \text{.}
\end{equation}

一つ目の文は「彼らは私をジョンと呼ぶ」という意味になり、二つ目の文は「私は私の部屋をキレイに保つ」という意味になる。
SVOCの構文をとる動詞は数が限られているので、頻出の形を覚えておけばよい。

\subsection{自動詞と他動詞}

後ろに目的語をとり、SVO・SVOO・SVOCの構文になる動詞を他動詞と呼び、後ろに目的語をとらず、SV・SVCの構文になる動詞を自動詞と呼ぶ。
まったく同じ動詞でも、他動詞としても自動詞としても使える単語が存在する。
どちらなのかで意味が変わることも多々あるため、動詞Vの直後が名詞の役割をするものなのかどうか、しっかり判断する必要がある。

\begin{equation}
  \ub{\us{代名詞}{I}}_S ~ \ub{\us{動詞}{run}}_V ~ \ub{\ur{\us{前置詞}{for} ~ \us{形容詞}{two} ~ \us{名詞}{hours}}_{副詞句}}_M \text{.}
\end{equation}

\begin{equation}
  \ub{\us{代名詞}{I}}_S ~ \ub{\us{動詞}{run}}_V ~ \ub{\us{冠詞}{a} ~ \us{名詞}{hospital}}_O \text{.}
\end{equation}

例えば$run$は、自動詞として使うと「走る」という意味になることが多いが、他動詞として使うと「経営する」という意味になることがある(もちろん$run$は多義語なので例外もあるが)。
よって、上の文は「私は2時間走る」という意味になるが、下の文は「私は病院を経営する」という意味になる。
単語帳で勉強するときは、自動詞としての意味と他動詞としての意味を把握しておくことが重要だ。
また、自動詞として使われるとき、後ろにどのような前置詞が続くのかも覚えておく必要がある。

\section*{Appendix}

\begin{table}[h]
  \centering
  \begin{tabular}{lll}
    \hline
    \multicolumn{1}{c}{日本語} & \multicolumn{1}{c}{英語(略称)} & \multicolumn{1}{c}{英語}\\
    \hline \hline
    \multicolumn{3}{c}{<品詞>}\\
    名詞 & N. & Noun\\
    代名詞 & Pron. & Pronoun\\
    動詞 & V. & Verb\\
    他動詞 & V.T. & Transitive Verb\\
    自動詞 & V.I. & Intransitive Verb\\
    助動詞 & Aux. & Auxiliary Verb\\
    形容詞 & Adj. & Adjective\\
    副詞 & Adv. & Adverb\\
    接続詞 & Conj. & Conjunction\\
    前置詞 & Prep. & Preposition\\
    冠詞 & & Article\\
    定冠詞 & & Difinite Article\\
    不定冠詞 & & Indifinite Article\\
    間投詞 & Interj. & Interjection\\
    \hline
    \multicolumn{3}{c}{<要素>}\\
    主語 & S & Subject\\
    述語 & V & Verb\\
    目的語 & O & Object\\
    間接目的語 & $\text{O}_1$, IO & Indirect Object\\
    直接目的語 & $\text{O}_2$, DO & Direct Object\\
    補語 & C & Complement\\
    修飾語 & M & Modifier\\
    \hline
  \end{tabular}
\end{table}

\end{document}