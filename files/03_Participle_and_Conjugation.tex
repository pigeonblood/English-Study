
\subsection{分詞}

分詞とは、動詞が活用し、名詞・形容詞・副詞といった他の品詞へと変化した形のことである。
ただし、単に分詞と言ったときは形容詞として使うケースを指す。
名詞として使うときは動名詞、副詞として使うときは分詞構文と呼ぶが、このあたりは後で扱う。
以下のような変化をした動詞は、もはや動詞ではなくなっているという点に注意されたい。

\begin{itemize}
  \item 現在分詞(Ving)\\
  動詞に$ing$がついた形で、形容詞や副詞として使われる。\\
  名詞として使われる場合のことを動名詞という。\\
  e.g. $do \rightarrow doing \text{,}~ make \rightarrow making \text{,}~ create \rightarrow creating$
  \item 過去分詞(Vpp)\\
  動詞に$ed$がついた形で、形容詞や副詞として使われる。\\
  現在分詞と違って不規則活用が多いので注意。\\
  e.g. $do \rightarrow done \text{,}~ make \rightarrow made \text{,}~ create \rightarrow created$
\end{itemize}

\begin{table}[h]
  \centering
  \begin{tabular}{ccc}
    \hline
     & 現在分詞 & 過去分詞\\
    \hline \hline
     名詞 & 動名詞 & \\
    形容詞 & 分詞 & 分詞 \\
    副詞 & 分詞構文 & 分詞構文 \\
    \hline
  \end{tabular}
\end{table}

形容詞としての分詞の例を以下に示す。
まずは現在分詞を使った例文である。

\begin{align}
  &\ub{\ur{\us{代名詞}{That} ~ \us{形容詞}{standing} ~ \us{名詞}{man}}_{名詞句}}_{S} ~ \ub{\us{動詞}{is}}_{V} ~ \ub{\us{代名詞}{my} ~ \us{名詞}{father}}_{C} \text{.}\\
  &\ub{\ur{\us{冠詞}{The} ~ \us{名詞}{man} ~ \ur{\us{形容詞}{standing} ~ \us{前置詞}{over} ~ \us{副詞}{there}}_{形容詞句}}_{名詞句}}_{S} ~ \ub{\us{動詞}{is}}_{V} ~ \ub{\us{代名詞}{my} ~ \us{名詞}{father}}_{C} \text{.}
\end{align}

現在分詞Vingは、「Vしている」と訳す。
今回は$standing$なので、「立っている」という意味になる。
上の例のように、形容詞を一単語だけ使う場合は、その形容詞はかかる名詞の前に置かれる(前置修飾)一方で、分詞に連れられて他の副詞などが存在し、それらで形容詞句を形成する場合は、形容詞句はかかる名詞の後に置かれる(後置修飾)。
端的に言えば、一単語なら名詞の前、複数単語なら名詞の後、ということである。

\begin{align}
  &\ub{\us{代名詞}{I}}_{S} ~ \ub{\us{動詞}{repair}}_{V} ~ \ub{\ur{\us{冠詞}{a} ~ \us{形容詞}{broken} ~ \us{名詞}{radio}}_{名詞句}}_{O} \text{.}\\
  &\ub{\us{代名詞}{I}}_{S} ~ \ub{\us{動詞}{repair}}_{V} ~ \ub{\ur{\us{冠詞}{a} ~ \us{名詞}{radio} ~ \ur{\us{形容詞}{broken} ~ \us{前置詞}{into} ~ \us{名詞}{pieces}}_{形容詞句}}_{名詞句}}_{O} \text{.}
\end{align}

過去分詞Vppは、「Vされた」と訳す。
今回は$broken$なので、「壊された」=「壊れた」という意味になる。
前置修飾と後置修飾のルールについても同様である。

\subsection{活用}

分詞以外での動詞の活用(品詞は動詞のまま)は、三人称単数現在と過去形の2パターンである。
これらの活用は、動詞に「役割を背負わせている」というイメージである。
その文章が過去を表すということを示すために、動詞に過去という役割を背負わせた結果、活用が起こるのだ。
ただし、助動詞が存在する場合は、助動詞が代わりにその役割を背負う。
よって、動詞はありのままの姿である原形に戻るのだ。

\begin{itemize}
  \item 三人称単数現在(三単現)\\
  主語が三人称かつ単数で、現在のことを表す文章では、動詞に$es$がつく。\\
  e.g. $do \rightarrow does \text{,}~ make \rightarrow makes \text{,}~ create \rightarrow creates$
  \item 過去形(Vp)\\
  過去のことを表す文章では、動詞に$ed$がつく。\\
  多くの場合で形自体は過去分詞と同じだが、不規則活用が多いので注意。\\
  e.g. $do \rightarrow did \text{,}~ make \rightarrow made \text{,}~ create \rightarrow created$
\end{itemize}

\subsection{be動詞の活用}

be動詞以外の動詞は、前段で説明したように、原形・三単現・過去形・現在分詞・過去分詞といった活用にとどまる。
しかし、be動詞は特殊な活用をするため、以下に表としてまとめておく。

\begin{table}[h]
  \centering
  \begin{tabular}{cllcc}
    \hline
    原形 & \multicolumn{1}{c}{現在形} & \multicolumn{1}{c}{過去形} & 現在分詞 & 過去分詞\\
    \hline \hline
     & 一人称単数 am & 一人称単数 was & & \\
    be & 三人称単数 is & 三人称単数 was & being & been \\
     & それ以外 are & それ以外 were & & \\
     \hline
  \end{tabular}
\end{table}

一般動詞は、原形と(三単現以外の)現在形は全く同じ形である。
しかし、be動詞は現在形でもすべて活用する。
では原形であるbeはどのようなときに出てくるのかというと、先述のように助動詞が出てくる場合である。
助動詞は動詞が背負っていた役割を奪って代わりに背負うため、be動詞は否応なしに原形であるbeに戻るのだ。
